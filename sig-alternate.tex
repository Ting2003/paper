\documentclass{sig-alternate}
\usepackage{enumerate}
\usepackage{verbatim}
\usepackage{url}
\usepackage{graphicx}
\usepackage{subfigure}
\usepackage{rotating}
\usepackage{cite}
\usepackage{multirow}
\def\thesection{\arabic{section}}
\graphicspath{{./figures/}}

\begin{document}
\title{Parallel Power Grid Simulation based on Multi-Core Platform}

%\author{Ting Yu, Zigang Xiao, Martin.D.F. Wong}
%	{Depart of Electrical and Computer Engineering, UIUC\\
%	Champaign IL 61820\\
%	Email: tingyu1@illinois.edu, xxxxxxxxx, mfwong@illinois.edu}
%}
\maketitle

\begin{comment}
This paper presents an efficient parallel Domain Decomposition method for large-scale power grid simulation. 
Based on multi-core platform, a new data storage strategy is proposed to overcome the memory bottleneck of traditional methods. 
Techniques as 3D irregular grid friendly overlapping, via detection as well as grouping techniques are utilized to accelerate the 
simulation. A new communication strategy is proposed and exhibits minimum communication overhead. Experimental results on several 
industrial and large-scale benchmarks show that the proposed method achieves more than 110X speedup over a state-of-the-art direct 
LU solver. Power grid containing over 190M nodes could be solved within 5 minutes.
\end{comment}

\begin{abstract}
Due to the development of CMOS process technology, power grid size is larger and larger. There are usually multi-million nodes in 
industrial power grid benchmarks. Analyzing the large size power grid has become expensive in both time and memory. This paper presents 
an efficient parallel iterative Domain Decomposition method for large-scale power grid simulation. Based on multi-core platform, a new 
data storage strategy is proposed to overcome the memory bottleneck of traditional methods. Much larger problem can be processed. 
Techniques as 3D irregular grid friendly overlapping, via detection as well as grouping techniques are utilized to accelerate the 
simulation. A new communication strategy is proposed and exhibits minimum communication overhead. Experimental results on several 
industrial and large-scale benchmarks show that the proposed method achieves more than 110X speedup over a state-of-the-art direct 
LU solver. It is the first time reported that power grid containing over 190M nodes could be successfully solved within 5 minutes.	
\end{abstract}

\section{Introduction}
Power grid analysis is an important process for consideration of signal integrity and further design 
optimization. To shorten the design period and cost, power grid simulation should be finished within a resonable amount of time. 
However, with the fast increase in VLSI circuit scale, nowadays, power grid usually contains millions or even hundreds of millions 
of nodes. The large problem size has made efficiently analyzing power grid a very challenging task. It would be prohibitively 
expensive for traditional solvers such as direct LU to solve the problem. Besides, traditional methods may suffer from memory 
bottle-neck, which makes them incapable to process large size problems. As a result, it is strongly required to develop an efficient 
solver to overcome these problems.
%\begin{comment}
	\begin{figure}[htbp]
	  \subfigure[][2D view of a power grid]{
		\includegraphics[width=0.2\textwidth]{2D_pg_1_new.pdf}\label{subfig.1}}
	  \subfigure[][3D view of power grid]{
		\includegraphics[width=0.25\textwidth]{3D_pg_new.pdf} \label{subfig.2}}
	  \caption{Simple model of power grid: 
	  \subref{subfig.1} 2D view,
	  \subref{subfig.2} 3D view.}
	  \label{pg_model}
	\end{figure}

	Figure~\ref{pg_model} shows a two layer simplified power grid model for DC analysis. Vias are used to 
	connect different layers. It is well known that analyzing the power grid equals to solving a linear system, which is 
%\end{comment}
	\begin{equation}\label{eq1}
		Gu=I
	\end{equation}
In the above equation, $G$ is the system conductance matrix, $u$ is the voltage solutions for each node, and $I$ is the current 
extracted from each node.

\begin{comment} 
	There are several serial simulation techniques developed to solve Equation \eqref{eq1}, including multigrid\cite{kozhaya}, 
	PCG\cite{Tsung-Hao}, random walks\cite{Boghrati}, domain-decomposition\cite{Quming,Zhongyu}, and so on. Random walks 
	is a stastical based method, and is only efficient for simulating a small subset of nodes, for eg., several thousand nodes.
	When it comes to the whole grid, it will either be time consuming to get solution, or will introduce error. The problems
	will deteriorate for larger size power grid with only a few number of VDD/GND sources. Besides, when modeling with vias, 
	this method can be trapped by vias, result in no convergence. PCG method in\cite{Tsung-Hao} is known as the fastest 
	linear solver 
	for large symmetric partial differential equations. But the total performance depends on the number of iterations. If the
	system is well-conditioned, only a few iterations are needed for convergence. In this case, due to larger effort per iteration
	for PCG than other simpler methods such as Gauss-sidel, as well as the fact that PCG consumes more memory, the total 
	performance of PCG may not be better than other simpler methods.
	
	Multigrid method achieves near linear simulation time with input size. The fine grid is recursively restricted into coarse ones 
	and the solution from coarse grids will be interpolated back into fine ones. Because of restriction and interpolation, 
	only approximate solution can be obtained. The error can be intractable for large or ill-conditioned system. Domain 
	decomposition is based on the idea of ``divide and conquer". 
	There are two kinds of domain decomposition(DD): Schur Complement DD (SCM) and Additive Schwarz DD (ASM). The main difference 
	is that Shur 
	Complement DD partitions find grid into subdomains with a common interface, while in Additive Schwarz DD, there is no 
	interface between subdomains. For SCM, 
	the dense characteristic of schur matrix limits the interface size to be small. As a result, this method can not solve too 
	large problems at low cost.
\end{comment}
	Considering the fact that power grid is usually in large size, even for an efficient serial solver, more and more time would be
	spent for simulation. When the input size is above some scale, the simulation time would be intrackable.
	Besides, the memory bottle-neck is also a problem. There will not be enough memory to hold the large size data. To address the
	simulation time issue, it should be seen that there is potential parallelism to solve this problem. For example, with "divide and
	conquer" based domain decomposition method, a lot of parallelism can be introduced during the simulation process. Combining the 
	fact that architectures as multi-core platform can overcome the memory bottle-neck, implementing a parallel method is a 
	promising way to achieve the efficient solver for large scale power grid analysis. 

	There are already several published work about analyzing power grid with parallel methods, including multigrid on 
	GPU\cite{Zhuofeng}, parallel direct LU 
	solver\cite{Super_LU_website}, public domain parallel ASM package\cite{PETSC_website}, and parallel DD
	\cite{kaisun, voronov}. Though GPU-based multigrid method\cite{Zhuofeng} is very fast, due to the interpolation and restriction
	process, error may be
	introducted in final solution, especially for ill-conditioned system. SuperLU\_MT and SuperLU\_DIST are two parallel direct LU 
	solvers developed in \cite{Super_LU_website}. SuperLU\_MT is 
	based on shared memory machine. Limited by the number of threads, not too much speedup can be gained. 
	It is reported to be only 5 to 10 folder faster than commecially popular solvers\cite{Super_LU_website}. Besides, the largest 
	problem this software can solve is limited by the system memory of the single processor/core. On the other hand, SuperLU\_DIST 
	is developed based on distribute memory machine. A single processor is used to read in the whole grid and build global 
	matrix, then distribute the subdomain matrices to other processors. 
	There is a obvious disadvantage for this strategy: the single processor wastes a lot of memory to store irrelevant data, as 
	it only needs to solve one subdomain. Due to this reason, the largest problem this package can 
	process can not be very large, which is reported to be 2 million nodes\cite{Super_LU_website}.

	Domain decomposition is very suitble for parallization, as there are many independent subdomains which can be
	solved in parallel in each iteration. PETSC\cite{PETSC_website} is such an implementation of ASM.
	However, there are several limitations. First, PETSC can only handle 2D regular grid. 
	The overlapping utilized is not supportive for 3D irregular grid. Another problem is that since a single processor is 
	still required to
	store the whole grid, it is not efficient to process very large size problem. Parallel domain decomposition in \cite{kaisun, voronov} have
	the same memory issue as PETSC.

	This paper proposes an efficient parallel implementation of ASM, which can not only process large scale problem size, 
	but also can achieve much speedup over a state-of-art direct LU solver. The main contributions include:
	\begin{enumerate}[1)]
	\item Based on distributed memory machine, a new data storage strategy is proposed to eliminate the memory bottle-neck of
	the above mentioned methods. Very large scale problem size can now be effeciently handled.
	\item A 3D irregular grid friendly 2D overlapping is implemented to help convergence. 
	\item A new data communication mechanism is proposed, which achieves minimum communication overhead.
	\item Via detection technique is presented to efficiently reduce matrix size without loss of accuracy.
	\item Grouping techinique is discussed to further increase parallelism.
	\end{enumerate}

	The rest of this paper is organized as follows. Section 2 gives an overview of ASM. Section 3 prensents 
	the proposed parallel ASM based on a distributed memory machine. Experimental results are illustrated in section 4. 
	Finally, conclusios are given in section 5.
%\begin{comment} 

\section{Additive Schwarz method (ASM) overview}	
	Additive Schwarz method (ASM) is a type of domain decomposition. It solves a boundary value problem for a partial 
	differential equation approximately by splitting it into boundary value problems on smaller domains and adding the results.
	There are two types of ASM: Algebraic ASM and geometric ASM. 
	Figure~\ref{Fig2}\subref{subfig.1} shows the algebraic ASM method utilized in \cite{kaisun}. The global matrix
	is partitioned into several submatrices in row direction. 
	Overlapping can be introduced at this stage. Jacobi or Gauss-sidel methods are then used to iterate among submatrix domains until 
	convergence is reached.
	\begin{figure}[htbp]
	  \subfigure[][Algebraic ASM]{
		\includegraphics[width=0.22\textwidth]{AASM_new.pdf}\label{subfig.1}}
	  \subfigure[][Geometric ASM]{
		\includegraphics[width=0.22\textwidth]{GASM_new.pdf} \label{subfig.2}}
	  \caption{Two types of ASMs: 
	  \subref{subfig.1} Algebraic ASM,
	  \subref{subfig.2} Geometric ASM.}
	  \label{Fig2}
	\end{figure}
	Figure~\ref{Fig2}\subref{subfig.2} illustrates the geomegtric ASM utilized in \cite{Zhongyu}. Subdomains are partitioned based on
	geometrical information. Overlapping can be introduced during this stage. Each subdomain then forms local matrix, which is
	solved in the same way as Algebraic ASM.
	In algebric ASM, overlapping can only happens between a subdomain and one neighboring subdomain. While in 
	geometric ASM, overlapping can be introduced between one subdomain and all its 8 neighboring subdomains.
	
%\end{comment}	
	It is known that introducing overlapping between different subdomains leads to faster convergence, 
	as overlapping reduces the matrix eigenspectrum\cite{Klawonn, Taopeng}. The more overlapping, the faster convergence.
	But the equivalent subdomain size will  be increased as well, which will deteriorate the performance. As a result, there is a 
	balance point of overlapping ratio to gain the optimal performance.
\begin{comment}
	SOR, PCG, MG methods can be combined with ASM to accelerate convergence. However, there are limitations of the resulting
	methods. SOR method is not 
	stable, as the optimal $w$ value depends on system matrix eigenvalues. It is hard to choose the optimal $w$ for large
	size problem, a bad $w$ value can even lead to divergence; MG may have
	error or even convergence problems, especially for 3D irregular grid; In PCG, a global
	matrix is required, which limits the maximum problem size that can be solved. 
\end{comment}
	There are already works talking about parallization of overlapping ASM, such as\cite{kaisun}. The are several problems in 
	\cite{kaisun}. First one is since each processor holds all the nodes' value of the whole grid, the memory bottleneck
	is not alleviated. Second, the communication strategy is not efficient: After each iteration, all the
	nodes' value will be gathered from all processors into a single processor, and broadcasted back to all processors. Much more
	data than needed is transferred. In fact, each processor only needs to exchange boundary nodes' value with its neighboring 
	subdomains. However, because of overlapping and 3D irregular grid structure, it is a challenging problem. Section 3
	gives a detail illustration of the situation.
 
\section{Parallel Implementation On \\Multi-core Processors}
  \subsection{Parallel platform}
	There are mainly two types of architectures for parallel processing: shared memory and distributed memory system. In shared 
	memory system, 
	multiple processors share common address space. There is no overhead for data communication. However, both the number
	of threads and memory size are limited. In distributed 
	memory system, a lot of processors are clustered together to form a local network. Each processor has its independent memory. 
	Extra time is needed for data communication. Considering the large problem 
	size and efficient simulation time requirement, distributed memory system is adopted in this paper. The implementation is based
	on widely utilized message-passing library(MPI).
  \subsection{Grid Partitioning}
	In \cite{Zhongyu}, 1D partitioning is thought to be in better performance than 2D partitioning. However, the comparison is not 
	fair, as the number of elements in 1D partitioned subdomain and 2D partitioned subdomain are not equal. Here we proves that
	for parallel consideration, 2D partitioning presents a better performance than 1D partitioning. 

%\begin{comment}	
	\begin{figure}[htbp]
	  \subfigure[][1D partitioning]{
		\includegraphics[width=0.22\textwidth]{1D_partition_new.pdf}\label{subfig.3.1}}
	  \subfigure[][2D partitioning]{
		\includegraphics[width=0.22\textwidth]{2D_partition_new.pdf}\label{subfig.3.2}}
	  \caption{2 types of partitioning:
	  \subref{subfig.3.1} 1D partitioning
	  \subref{subfig.3.2} 2D partitioning }
	  \label{Fig3}
	\end{figure}
	Figure~\ref{Fig3} lists the two types of partitioning. For fairness, each subdomain in 1D partitioning contains the 
	same number of nodes with 2D partitioning. As a result, the total number of subdomains for 1D partitioning and 2D partitioning are the same. For 
	each subdomain 
	in Figure~\ref{Fig3}\subref{subfig.3.2}, assume  it is square and each side contains $k$ nodes. The total 
	number of nodes' value needed to be exchanged with neighboring subdomains would be $4k$. On the other hand, as 
	there are $m_2$ subdomains along vertical direction, for a subdomain in Figure~\ref{Fig3}\subref{subfig.3.1}, each side contains
	$m_2k$ nodes. As a result, $2m_2k$ nodes' value are needed to be exchanged with its left and right 
	neighboring subdomains. Equation\eqref{eq2} illustrates the communication cost relationship between these two types of
	partitioning. As $m_2>2$ is generally always true, 2D partitioning has more favorable surface-to-volume ratio.    
	\begin{equation}
		2m_2k > 4k, \hspace{0.5 in} m_2 > 2 \label{eq2}
	\end{equation}
%\end{comment}
	Besides, 2D partitioning presents a better scalability than 1D partitioning. This can be
	seen from the isoefficiency expressions of 2 cases. Isoefficiency represents the scalability of algorithm
	with input size. The lower order of isoefficiency expression, the better scalabilify. Isoefficiency functions for 1D 
	and 2D partitioning are respectively $\Theta(p^2)$ and $\Theta(p)$\cite{Grama, Kumar}, where p is the number of processors. 
	For given number of processors, far smaller problem is required for 2D partitioning to achieve the given efficiency.

	Based on above considerations, 2D partitioning is utilized in this paper. 
  \subsection{Data storage}
\begin{comment}
	In parallel program, it is very important to keep data balanced among processors. This is not only true for computing
	consideration, but also for memory storage consideration. 
\end{comment}
	In traditional multi-core parallel programs, input data is first loaded into a single processor 
	and distribute to others. The single processor spends a lot of extra memory to store the whole grid information,
	which limits the maximum problem size that can be solved. In this paper, we propose a new data storage strategy to avoid the above problem.
	With this strategy, vary large scale problem size can be processed.

	According to the number of subdomains and the grid's geometrical information, each subdomain's boundary information can be 
	extracted. With this information, input netlist is divided into several different files, where each file maintains all the net 
	information of a subdomain. This process is finished by a single processor. Then, each processor parses data from its
	corresponding file and build up local matrix simultaneously.

	Though writing I/O process is slow, the parsing stage, which is the main time consuming part and is 
	processed by a single processor in sequential method, can now be executed in parallel. As a result, 
	there will not be extra overhead with this strategy. 
     \subsection{Communication overhead}
	Local matrices of subdomains are solved by direct solvers. Gauss-Sidel iteration method is used
	to update different subdomains until convergence. In the end of each iteration, each subdomain needs to update its 
	boundary nodes' value. With overlapping, there are two types of boundary nodes: Nodes that do no 
	belong to a subdomain, but are connected to internal nodes of 
	this subdomain; nodes that belong to a subdomain and are boundary nodes of neighboring subdomains. The 
	two types of boundary nodes are illustrated in Figure~\ref{comm}. In the rest of this paper, the first type of nodes are defined
	``outer boundary nodes", while second are defined ``inter boundary nodes". 

	An intuitive strategy for sending and collecting boundary information would be as follows:

	\begin{enumerate}[1.]
	  \item Each subdomain sends its inter boundary nodes' value sequentially to 8 neighboring subdomains;
	  \item The subdomain then collects its outer boundary nodes' value sequentially from 8 neighboring subdomains.
	\end{enumerate}	

 	However, this strategy is not efficient in message passing system. The reason can be seen from analyzing the
	communication model:
	\begin{equation}
		T_{msg} = t_s + t_wL\label{eq_msg}
	\end{equation}
	where $t_s$ is the start up time, $t_w$ is the incremental transfer time per word, and $L$ is the length of message in words.
	
	Generally, $t_s$ is tens of thousands larger than $t_w$. It is highly perferable to send less 
	large messages rather than many small messages to reduce the start up time. Since the total number of boundary nodes is small, 
	time spend on $t_wL$ is short. For the above mentioned strategy, $8\times 2 = 
	16$ sending or receiving operations are required per iteration. As a result, in total $16t_s$ start up time is
	needed per iteration. As a lot of iterations may be required beforing reaching convergence, the total 
	start up time should not be ignored. This overhead can be optimized by the new strategy shown in the following paragraphs.
\begin{comment}
	In each iteration, only $2$ transferring operations are needed for communication. The new strategy minimizes start up time and 
	greatly decreases the simulation time. However, there is some difficulty to implement above strategy. This is not only because
	the grid connection is irregular, but also because the existance of overlapping. Overlapping makes things even more complicated, 
	as the number of neighboring subdomains are increased from $4$ to $8$, with different overlapping patterns. To overcome this 
	problem, this paper proposes a irregular grid and overlapping friendly communication strategy that achieves optimal 
	minimum communication overhead, which is illustrated as follows.
\end{comment}
%\begin{comment}
	\begin{figure}[htbp]
	  \centering
	  \includegraphics[width=0.45\textwidth]{communication_1_new.pdf}
	  \caption{Boundary nodes distribution}
	  \label{comm}
	\end{figure}
%\end{comment}

	Figure~\ref{comm} shows the boundary nodes' distribution along vertical direction. Plain blocks refer to subdomains without 
	overlapping, while shadowed blocks refer to subdomains with some extent of overlapping. There are several difficulties in 
	developing an efficient communication strategy with this model. As we can see, the net connection can be irregular. Further, 
	as power grid is in multilayer 
	structure, the inter and outer boundary nodes will appear at the same or different positions
	among different layers. Moreover, because of overlapping, there can be multiple solutions for a node's value at the boundary 
	overlapping area, as this node belongs to several subdomains at the same time. For convergence consideration, only one value 
	should be used to update subdomains. Here we use the node's value from the subdomain that contains the node and has the biggest 
	index.

	To overcome the above difficulties, it is observed that geometrical information can be utilized. Based on the geometrical 
	characteristic of inter and outer boundary nodes, we store each type of node into 8 node sets. Each subdomain has 16 node sets. 
	For example, for subdomain 5, ``south east" node set of its outer boundary array stores internal boundary nodes' value of 
	south east neighboring subdomain, which is subdomain 3, while ``south east" node set in inter boundary array stores outernal 
	boundary nodes' value of south east neigboring subdomain, which is subdomain 7. Boundary nodes within 
	these node sets are sorted according to their 3D coordinates, so that they are insensitive to 3D irregular structure. 

	Each node will be
	indexed according to its ordering in the sorted boundary node sets. As the node sets are stored according to relative geometrical 
	relationship between 2 subdomains, it is easy to access boundary nodes' value.
	For example, in Figure~\ref{comm}, inter boundary nodes along ``west north" 
	direction of subdomain 5 are also the outer boundary nodes along ``south east" direction of subdomain 7. Each node in the 
	2 node sets shares the same index. With this relationship, boundary nodes' value can be efficiently obtained.

	To minimize the start up time, for each processor, we allocate a local inter and outer boundary array, to store the 16 boundary
	node sets' values. Each array stores 8 corresponding node sets' values in the increasing order of neighboring subdomains' index. 
	After each iteration, each subdomain has new solutions for all internal nodes. The inter 
	boundary nodes' value will then be copied into local inter boundary array. Processor 0 gathers these arrays from each subdomain
	into a global inter boundary array. Then, this array will be reordered to generate the global outer boundary array. This array
	contains all the outer boundary nodes' value for each subdomain. 

	The reordering process is shown in Figure~\ref{Fig4}. Labels in global inter boundary array indicates the subdomain index, where
	the local inter boundary array of corresponding subdomain is stored. Each local inter boundary array contains 8 elments, storing
	the 8 node sets. The labels within the local
	inter boundary arrays represents ``<subdomain\_id>\_<boundary\_node\_set\_direction>".

	To gather all the updated outer boundary nodes' values of a subdomain, each outer boundary node set's values needs to be located
	and copied from corresponding internal boundary node set position of neighboring subdomains. For example, the 
	outer boundary array of subdomain 5 includes the ``north west" inter boundary node set of subdomain 1, ``north" inter boundary 
	node set of 
	subdomain 2, and so on. We use processor 0 to scan a subdomain's neighboring subdomains by increasing index and locates the 
	local inter boundary node set. Their values are copied into corresponding places in global outer boundary array. The 
	reordered global array will then be scattered to all processors, so that each processor has its outer boundary nodes' value 
	ready for later operation. 
%\begin{comment} 
	\begin{figure}[htbp]
	  \centering
	  \includegraphics[width=0.45\textwidth]{reordering_new.pdf}
	  \caption{Reorder inter boundary array into outer boundary array}
	  \label{Fig4}
	\end{figure}
%\end{comment}

	After a subdomain receiving its local outer boundary array, the values are assigned to corresponding internal boundary 
	nodes. As 
	mentioned before, nodes in overlapping area have several different values from overlapping subdomains. For these nodes, 
	as their values are assigned according to the ascending order of neighboring subdomain, the new value will always cover old one. 
	After the assigning process, the node's value is from the largest index neighboring subdomain that contains this node.  
	
	In the above method, only one gathering and one scattering operations are needed per iteration. As a result, the start up time 
	would only be $2t_s$, which is much less than the previously mentioned $16t_s$. In fact, it is also the minimum 
	start up time for this problem. At the same time, since the total number 
	of boundary nodes are always very small, combining with the fact that $t_w$ is very small, it can be concluded that the 
	time for transfering messages is negligible. 

	In conclusion, the proposed strategy sucessfully achieves munimum communication overhead. It is insensitive to 3D irregular 
	grid structure, and also effectively overcomes the difficulties caused by overlapping.	
  \subsection{Via detection technique}
	As mentioned before, power grid nodes are vertically connected by vias. If the vias are with no 
	resistance, it is called "shortened", and those nodes always have identical voltage value. Because of this relationship, 
	only stamping the representive
	node into matrix is a effective method to reduce the matrix size. In extreme case, if all the
	vias are shortened, 3D power grid can be modeled as a 2D grid without any loss of accuracy. The resulting matrix size
	is much smaller, and a lot of simulation time can be saved. 
 
	It is also observed that if the vias connecting nodes have very small resistance value, it can also be treated as ``shortened" vias 
	without any loss of accuracy. We set a threshold value for via nets. Via nets 
	whose values are below this threshold will be substitued with shortened nets. Without loss of accuracy, this threshold value 
	is experimentally set to be 1e-5. 
  \subsection{Grouping technique} 
	Power grid usually contains several physically disjoint components, for example, VDD and GND sub-circuit. With multi-voltage 
	technique, more components can be included, for example, VDD2 circuit. Since these components are totally independent, they can
	be solved in parallel. We utilize the group collective 
	communication technique of MPI to implement this idea. Processors are grouped into several clusters, and each cluster 
	simulates one component. 
	If all the components are similar in size, half or even more runtime can instantly being saved. 
	It should be noticed that the time saving comes at a cost of using more processors. If $p$ processors are used to 
	solve different components sequentially, for a benchmark of $n$ components, the grouping technique requires $np$ processors to
	do the simulation.  
\section{Numerical Experiments}
	We implement a direct LU solver: UMFPACK on a single CPU, and the proposed ASM on multi-core platform. 
	All codes are written in C++ and compiled with mpicxx under Linux system. 
  \subsection{Simulation Environment}
	The single CPU is with 2.67 GHz frequency and 24 GB RAM. The parallel platform is composed of a set of cluster 
	computing nodes, each with two 2.67 GHz Intel Xeon hex-core processors. Each node has in total 24 GB of RAM, with about 
	1.8 GB memory per processor. Message passing package MPICH2 is used.  
  \subsection{Implementation}
	Error tolerance is set to be $10^{-5}$ for all test cases in order to meet the following accuracy requirement:
	\begin{equation}
		Max(e) < 0.01 mv, Avg(e) < 0.1 mv\\ \label{eq_err}
	\end{equation}
	In the experiment, overlapping ratio is defined to be the ratio between overlapping length to the total side length of 
	a subdomain. To achieve a good balance between iteration numbers and matrix size, overlapping ratio is experimentally 
	setted to be 0.2.
			
	Three sets of benchmarks are used for comparison, including 
	industrial benchmarks pg3 to pg6, x250, x200, y250, y200 and 4 artifially produced benchmarks. pg3 to pg6 are 
	small size wire bond type benchmarks with vias, x250, x200 and y250, y200 are respectively two median C4 type benchmarks with
	shortened vias or regular vias. The 4 large scale benchmarks are all C4 type grids with vias. Simulation results are 
	listed in Table \ref{tb_v}, \ref{tb_s}, \ref{tb_m}, and \ref{tb_l}.
	\begin{table}[h]
	  \include{./tables/table_viagroup} \label{tb_v}
	%\end{table}	
	%\begin{table}[h]
  	  \include {./tables/table_small} \label{tb_s}
	\end{table}
	\begin{table}[h]
  	  \include{./tables/table_median} \label{tb_m}
	%\end{table}
	%\begin{table}[h]
  	  \include{./tables/table_large} \label{tb_l}
	\end{table}

	The meanings of items in the tables are as follows: \#N: total number of nodes in the circuit; 
	\#B: number of blocks; \#C: number of disjoint components; \#Rep: number of replist nodes; 
  	$t_N$: simulation time with original nodes; $t_{Rep}$: simulation time with replist nodes; 
	$t_g$: simulation time with replist nodes and grouping technique; $t_c$: simulation time for direct LU solver. 
	$t_p$: simulation time with proposed method in this paper.  
	All $t_c$ is obtained by solving the circuits with independent components. All 
	simulation time includes both grid set up session and solve session.

	Data in Table \ref{tb_s} demonstrates the effectiveness of via detection technique and grouping technique. With via detection 
	technique, the 
	number of nodes which will be stamped into matrix is decreased by nearly 2/3. Comparing $t_N$ and $t_{Rep}$, we can see there is a
	great reduce in simulation time. Adding grouping technique decreases the simulation time further by half. If the 3 components are 
	similar in size, even shorter simulation time can be achieved.

	Table \ref{tb_s}, \ref{tb_m} exibits the obvious advantage of the proposed method. There is a tremendous 
	speedup over direct LU solver. For each benchmark, there is at least tens of speedup. The biggest speedup is over 110X. 
	%%%%% need to add more data nodes to show the limitation of over much processors.
	%If more processors can get allocated, it is possible to further decrease the runtime.\\

	Table \ref{tb_l} shows that the proposed method sucessfully finishes the simulation of large scale power grids
	in a short time, while LU can not handle the problem, due to its large size. In the experiment, the largest problem that LU can solve is 
	around 8M nodes per component. 

	It should be noticed that since each processor in the multi-core system has smaller memory size, grid needs to be 
	partitioned into enough subdomains so that each subdomain is small enough to be load in memory. As one processor handles one 
	subdomain, enough processors will be needed. For large scale benchmarks, if there is sufficient processor resource, several 
	components can be solved simultaneously. If there is not enough processors, those components should be 
	solved sequentially. In Table \ref{tb_l}, netlist\_32M, netlist\_72M and netlist\_108M are 
	simulated with components being solved simultaneously. Each component consumes 500 processors. For netlist\_144M and 
	netlist\_192M, as not enough processors can get allocated, components are solved sequentially.

\section{Conclusions}
	This paper presents detail implementation of an efficient parallel ASM for DC power grid analysis. Based on distribute memory 
	system, a new data storage strategy is proposed to efficiently handle the large problem size. 3D irregular grid 
	friendly overlapping 
	is introduced to accerelate the convergence. A new method is presented to achieve the minimum communication overhead. Via 
	detection and grouping technique are adopted to decrease the problem size, and introduce more parallelism. Experimental 
	results of several industrial benchmarks show that more than 110X speedup are gained over a state-of-art LU solver. It is also 
	the first time reported that benchmark as large as 192M can be processed and finished within 5 minutes. With enough processors, 
	even larger benchmarks can be simulated.

%\section*{Acknowledgment}

\bibliographystyle{abbrv}
\bibliography{sig-alternate}
\end{document}
