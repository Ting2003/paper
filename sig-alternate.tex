\documentclass{sig-alternate}
\usepackage{enumerate}
\usepackage{verbatim}
\usepackage{graphicx}
\usepackage{subfigure}
\usepackage{rotating}
\usepackage{cite}
\usepackage{multirow}
\def\thesection{\arabic{section}}
\graphicspath{{./figures/}}

\begin{document}
\title{Parallel Power Grid Simulation based on Multi-Core Platform}

%\author{Ting Yu, Zigang Xiao, Martin.D.F. Wong}
%	{Depart of Electrical and Computer Engineering, UIUC\\
%	Champaign IL 61820\\
%	Email: tingyu1@illinois.edu, xxxxxxxxx, mfwong@illinois.edu}
%}
\maketitle

\begin{abstract}
	Due to the development of deep submicron technique, powergrid now are becoming denser and denser. There are usually 
	multi-million nodes in industry powergrid benchmarks. Analyzing power grid has become computationnally 
	intrackable and memory overconsuming. Based on multi-core platform, this paper presents an effecient parallel iterative 
	Additive Schwarz Domain Decomposition (ASM) for large-scale power grid simulation. A new data storage strategy is proposed 
	to overcome
	the disadvantage of traditional domain decomposition, where a single processor is required to store the whole grid information.
	As a result, much larger problem size can be handled now. Techniques as irregular and multilayer grid friendly full 2D 
	overlapping, via detection as well as grouping technique are utilized to accelerate the simulation. The prpposed new
	communication strategy exibits optimal minimum communication overhead. Experiment results on several industrial and large 
	scale benchmarks show that the proposed method 
	achieves more than 110X speedup over a state-of-the-art direct LU solver. Besides, it is the first time 
	reported in this paper that power grid contains over 190M nodes can be sucessfully solved within less than 5 minutes.
\end{abstract}

\section{Introduction}
	For consideration of signal integrity, verification and further design optimization, powergrid analysis is one of the most 
	important steps. To shorten the design period and cost, power grid simulation should be finished within a resonable amount
	of time. 
	However, with the fast increase of VLSI circuit scale, nowadays, power grid usually contains millions or even hundreds 
	of millions of nodes. The large problem size has made efficiently analyzing power grid a very challenging task. 
	It would be prohibitively expensive for traditional solvers such as direct LU to solve the problem. Besides, traditional 
	methods may suffer from memory bottle-neck, which makes them incapable to process the large size problems. As a result, 
	it is strongly required to develop an efficient solver to overcome these problems.\\
%\begin{comment}
	\begin{figure}[htbp]
	  \subfigure[][2D view of a power grid]{
		\includegraphics[width=0.2\textwidth]{2D_pg_1_new.pdf}\label{subfig.1}}
	  \subfigure[][3D view of power grid]{
		\includegraphics[width=0.27\textwidth]{3D_pg_new.pdf} \label{subfig.2}}
	  \caption{Simple model of power grid: 
	  \subref{subfig.1} 2D view,
	  \subref{subfig.2} 3D view.}
	  \label{pg_model}
	\end{figure}

	Figure~\ref{pg_model} shows a two layer simplified power grid model without capacitor or inductor. Vias are used to 
	connect different layers. If transient analysis is performed, RLC model should be used.\\ 

	It is well known that analyzing the power grid equals to solving a linear system, which is 
%\end{comment}
	\begin{equation}\label{eq1}
		Ax=b
	\end{equation}
	There are several serial simulation techniques developed to solve Equation \eqref{eq1}, including multigrid\cite{kozhaya}, 
	PCG\cite{Tsung-Hao}, random walks\cite{Boghrati}, domain-decomposition\cite{Quming,Zhongyu}, and so on. Random walks 
	is a stastical based method, and is only efficient for simulating a small subset of nodes, for eg., several thousand nodes.
	When it comes to the whole grid, it will either be time consuming to get solution, or will introduce error. The problems
	will deteriorate for larger size power grid with only a few number of VDD/GND sources. Besides, when modeling with vias, 
	this method can be trapped by vias, result in no convergence. PCG method in\cite{Tsung-Hao} is known as the fastest 
	linear solver 
	for large symmetric partial differential equations. But the total performance depends on the number of iterations. If the
	system is well-conditioned, only a few iterations are needed for convergence. In this case, due to larger effort per iteration
	for PCG than other simpler methods such as Gauss-sidel, as well as the fact that PCG consumes more memory, the total 
	performance of PCG may not be better than other simpler methods.\\	
	
	Multigrid method achieves near linear simulation time with input size. The fine grid is recursively restricted into coarse ones 
	and the solution from coarse grids will be interpolated back into fine ones. Because of restriction and interpolation, 
	only approximate solution can be obtained. The error can be intractable for large or ill-conditioned system. Domain 
	decomposition is based on the idea of ``divide and conquer". 
	There are two kinds of domain decomposition(DD): Schur Complement DD (SCM) and Additive Schwarz DD (ASM). The main difference 
	is that Shur 
	Complement DD partitions find grid into subdomains with a common interface, while in Additive Schwarz DD, there is no 
	interface between subdomains. For SCM, 
	the dense characteristic of schur matrix limits the interface size to be small. As a result, this method can not solve too 
	large problems at low cost.\\

	Considering the fact that power grid is usually in large size, even for an efficient serial solver, more and more time would be
	spent for simulation. When the input size is above some scale, the simulation time would be intrackable.
	Besides, the memory bottle-neck is also a problem for single processor. On the other hand, there is potential parallelism with
	some algorithms to solve this problem, for example, DD. Combining the fact that architectures as multi-core platform 
	can overcome the memory bottle-neck, implementing a parallel method is a promising way to achieve a efficient solver for 
	large scale power grid analysis.\\ 

	There are already several published work about analyzing power grid with parallel methods, including multigrid on GPU\cite{Zhuofeng}, parallel direct LU 
	solver\cite{Super_LU_website}, public domain parallel ASM package\cite{PETSC_website}, and parallel DD
	\cite{kaisun, voronov}. Though GPU-based multigrid method\cite{Zhuofeng} is very fast, the above mentioned problems with 
	multigrid method 
	still exist. SuperLU\_MT and SuperLU\_DIST are two parallel direct LU solver developed in \cite{Super_LU_website}. SuperLU\_MT is 
	based on shared memory machine. Limited by the number of threads, not too much speedup can be gained. 
	It is reported to be only 5 to 10 folder faster than commecially popular solvers\cite{Super_LU_website}. Besides, the largest 
	problem this software can solve is limited by the system memory of the single processor/core. On the other hand, SuperLU\_DIST 
	is developed based on distribute memory machine. A single processor is used to first read in the whole grid and build global 
	matrix, then distribute the subdomain matrices to other processors. 
	There is a obvious disadvantage for this strategy: the single processor wastes a lot of memory to store irrelevant data, as 
	it only needs to solve one subdomain. The problem this package can 
	process can not be very large as well. The largest problem size that is reported for this package to solve is 2 million nodes
	\cite{Super_LU_website}.\\

	Domain decomposition is very suitble for parallization, as there are many independent subdomains which can be
	solved in parallel. PETSC\cite{PETSC_website} is such an implement of ASM.
	However, there are several limitations. First, PETSC can only handle regular and uniform grid. It is only fit for 2D grid. 
	The overlapping utilized is not irregular and multilayer grid friendly. Another problem is that since a single processor is 
	still required to
	store the whole grid, it is not fit for very large size problem. Parallel domain decomposition in \cite{kaisun, voronov} have
	the same memory issue as PETSC.\\

	This paper proposes an efficient parallel implementation of ASM, which can not only process large scale problem size, 
	but also can achieve much speedup over a state-of-art direct LU solver. The main contributions include:
	\begin{enumerate}[1)]
	\item Based on distributed memory machine, a new data storage strategy is proposed to eliminate the memory bottle-neck of
	the above mentioned methods. Very large scale problem size can now be effeciently handled.
	\item A irregular and multilayer grid friendly full 2D overlapping is implemented to help convergence. 
	\item A new data communication mechanism is proposed, which achieves optimal minimum communication overhead.
	\item Via detection technique is presented to efficiently reduce matrix size without loss of accuracy.
	\item Grouping techinique is introduced to further increase parallelism.
	\end{enumerate}

	The rest of this paper is organized as follows. Section 2 gives an overview of ASM. Section 3 prensents 
	the proposed parallel ASM based on a distributed memory machine. Experimental results are illustrated in section 4. 
	Finally, conclusion is given in section 5.
%\begin{comment} 

\section{Additive Schwarz method(ASM) overview}	
	Named after Hermann Schwarz, ASM solves a boundary value problem for a partial 
	differential equation approximately by splitting it into boundary value problems on smaller domains and adding the results.
	There are two types of ASM: Algebraic ASM and geometric ASM. 
	Figure~\ref{Fig2}\subref{subfig.1} shows the algebraic ASM method utilized in \cite{kaisun}. The global matrix
	is partitioned into several subdomains in row direction. 
	Overlapping can be introduced at this stage. Jacobi or Gauss-sidel methods are then used to iterate among subdomains until 
	convergence is reached.\\	
	\begin{figure}[htbp]
	  \subfigure[][Algebraic ASM]{
		\includegraphics[width=0.22\textwidth]{AASM_new.pdf}\label{subfig.1}}
	  \subfigure[][Geometric ASM]{
		\includegraphics[width=0.22\textwidth]{GASM_new.pdf} \label{subfig.2}}
	  \caption{Two types of ASMs: 
	  \subref{subfig.1} Algebraic ASM,
	  \subref{subfig.2} Geometric ASM.}
	  \label{Fig2}
	\end{figure}

	Figure~\ref{Fig2}\subref{subfig.2} illustrates the geomegtric ASM utilized in \cite{Zhongyu}. Subdomains are partitioned based on
	geometrical information. Overlapping can be introduced during this stage. Each subdomain then forms local matrix.
	Different from algebric ASM, where overlapping can only happens between a subdomain and one neighboring subdomain, in 
	geometric ASM, full 2D overlapping can be introduced between one subdomain and all its 8 neighboring subdomains, as is seen from 
	Figure~\ref{Fig2}\subref{subfig.2}.\\
	
%\end{comment}	
	It is known that introducing overlaping between different subdomains leads to faster convergence, 
	as overlapping reduces the matrix eigenspectrum\cite{Klawonn, Taopeng}. The more overlapping, the faster convergence.
	But the equivalent subdomain size will  be increased as well. As a result, there is a balance point of overlapping ratio to 
	gain the optimal performance.\\

	SOR, PCG, MG methods can be combined with ASM to accelerate convergence. However, there are limitations of the resulting
	methods. SOR method is not 
	stable, as the optimal $w$ value depends on system matrix eigenvalues. It is hard to choose the optimal $w$ for large
	size problem, a bad $w$ value can even lead to divergence; MG may have
	error or even convergence problems, especially for 3D irregular grid; In PCG, a global
	matrix is required, which limits the maximum problem size that can be solved.\\ 
	 
	There are already works talk about parallization of overlapping ASM, such as\cite{kaisun}. One problem in \cite{kaisun} is 
	that the
	data load is not evenly distributed. Also, the communication strategy is not efficient: Each processor stores all the nodes' 
	value. After each iteration, all the
	nodes' value will be gathered from all processors into a single processor, and broadcasted back to all processors. Since
	only boundary nodes' values of corresponding subdomain are all the information that is needed for processors, and the number of 
	boundary nodes is only a very small portion of the whole grid, it would be a huge waste to transfer all nodes' value 
	after each iteration. Each processor only need to exchange boundary nodes' value with its neighboring subdomains.
	However, because of overlapping and irregular grid structure, it is a challenging problem. Section 3
	gives a detail illustration of the situation.
 
\section{Parallel Implementation On \\Multi-core Processors}
  \subsection{Parallel platform}
	There are two types of multi-core architectures: shared memory and distributed memory machine. In shared memory machine, 
	multiple processors share common address space, there is no overhead for data communication. However, both the number
	of threads and memory size are limited. In distributed 
	memory machine, a lot of processors can be clustered, where each processor has its independent memory. Extra time 
	is needed for data communication. Considering the large problem 
	size and efficient simulation time requirement, distributed memory machine is adopted in this paper. The implementation is based
	on widely utilized message-passing library(MPI).
  \subsection{Grid Partitioning}
	In \cite{Zhongyu}, 1D partitioning is thought to be in better performance than 2D partitioning. However, the comparison is not 
	fair, as the number of elements in 1D partitioned subdomain and 2D partitioned subdomain are not equal. Here we proves that
	for parallel consideration, 2D partitioning presents a better performance than 1D partitioning. It is because 
	partitioning across more dimensions yields more neighboring subdomains but smaller total volume of communication.\\

%\begin{comment}	
	Suppose the grid is partitioned in 1D subdomains and 2D subdomains, as shown in Figure~\ref{Fig3}.
	\begin{figure}[htbp]
	  \subfigure[][1D partitioning]{
		\includegraphics[width=0.22\textwidth]{1D_partition_new.pdf}\label{subfig.3.1}}
	  \subfigure[][2D partitioning]{
		\includegraphics[width=0.22\textwidth]{2D_partition_new.pdf}\label{subfig.3.2}}
	  \caption{2 types of partitioning:
	  \subref{subfig.3.1} 1D partitioning
	  \subref{subfig.3.2} 2D partitioning }
	  \label{Fig3}
	\end{figure}

	For fairness, each subdomain in 1D partition and 2D partition has the same number
	of nodes. As a result, the total number of subdomains for 1D partitioning and 2D partitioning are the same. For each subdomain 
	in Figure~\ref{Fig3}\subref{subfig.3.2}, suppose it is square and each side is contains $k$ nodes. As a result, the total 
	number of nodes' value needed to be exchanged with neighboring subdomains would be $4k$. On the other hand, as 
	there are $m_2$ subdomains along vertical direction, for a subdomain in Figure~\ref{Fig3}\subref{subfig.3.1}, each side contains
	$m_2k$ nodes. As a result, there are $2m_2k$ nodes' value needed to be exchanged with its left and right 
	neighboring subdomains. Equation\eqref{eq2} illustrates the relationship of the communication load between these two types of
	partitioning. As $m_2>2$ is generally always true, 2D partitioning thus has more favorable surface-to-volume ratio.    
	\begin{equation}
		2m_2k > 4k, \hspace{0.5 in} m_2 > 2 \label{eq2}
	\end{equation}
%\end{comment}
	Besides, considering parallel performance, 2D partitioning presents a better scalability than 1D partitioning. This can be
	illustrated from the isoefficiency expressions of 2 cases. Isoefficiency represents the scalability of algorithm
	with input size. The lower order of isoefficiency expression, the better scalabilify. Isoefficiency functions for 1D 
	and 2D partitioning are respectively $\Theta(p^2)$ and $\Theta(p)$\cite{Grama, Kumar}, where p is the number of processors. 
	For given number of processors, far smaller problem is required for 2D partitioning to achieve given efficiency.\\

	Combining the above considerations, 2D partitioning is utilized in this paper.\\	
	  
  \subsection{Data storage}
	In parallel program, it is very important to keep data balanced among processors. This is not only true for computing
	consideration, but also for memory storage consideration. In \cite{Quming}, input data is first loaded into a single processor 
	and distribute to others. As the single processor spends a lot of extra memory to store the whole grid information,
	the problem size that can be solved is limited. In this paper, we propose a new data distribution method, where each processor
	only needs to store the subdomain information. With this strategy, vary large scale problem size can now be processed.\\

	According to the number of subdomains and the grid's geometrical information, each subdomain's boundary information can be 
	extracted. With this information, input netlists are concluded into different files, where each file maintains all the nets 
	belongs	to a subdomain. This process is finished by a single processor. Then, each processor parses data from 
	corresponding file and build up local matrix simultaneously.\\

	Though writing files process is slow, the parsing stage, which is the main time consuming part and is 
	processed by a single processor in sequential method, can now be executed by all the processors in parallel. As a result, 
	there will not be extra overhead with this strategy.\\
	 
     \subsection{Communication overhead}
	Local matrices of subdomains are solved by direct solvers. Gauss-Sidel iteration method is used
	to update different subdomains until convergence is reached. After each iteration, each subdomain needs to update its 
	boundary nodes' value. With overlapping, there are two types of boundary nodes: First type nodes are those that do no 
	belong to a subdomain, but are connected to internal nodes of 
	this subdomain; Second type nodes are those that belongs to a subdomain and are boundary nodes of neighboring subdomains. The 
	two types of boundary nodes are illustrated in Figure~\ref{comm}. In the rest of this paper, the first type of nodes are defined
	``outer boundary nodes", while second type of nodes are defined ``inter boundary nodes".\\ 

	An intuitive strategy for sending and collecting boundary information would be as follows:

	\begin{enumerate}[1.]
	  \item A subdomain sends its inter boundary nodes' value sequentially to 8 neighboring subdomains;
	  \item This subdomain then collects its outer boundary nodes' value sequantially from 8 neighboring subdomains.
	\end{enumerate}	

 	However, this strategy is not efficient in message passing system. The reason can be seen from analyzing
	communication model:
	\begin{equation}
		T_{msg} = t_s + t_wL\label{eq_msg}
	\end{equation}
	where $t_s$ is the start up time, $t_w$ is the incremental transfer time per word, and $L$ is the length of message in words.\\
	
	Generally, $t_s$ is tens of thousands larger than $t_w$. It is highly perferable to send less 
	large messages rather than many smalle messages to reduce the start up time. Since the total number of boundary nodes is small, 
	time spend on $t_wL$ is short. For the above mentioned strategy, $8\times 2 = 
	16$ sending or receiving operations are required for a subdomain per iteration. As a result, in total $16t_s$ 
	would be needed for starting up per iteration. As a lot of iterations may required beforing reaching convergence, the total 
	start up time should not be ignored. This overhead can be optimized minimized by taking another strategy.
	In each iteration, only $2$ transferring operations are needed for communication. The new strategy minimizes start up time and 
	greatly decreases the simulation time. However, there is some difficulty to implement above strategy. This is not only because
	the grid connection is irregular, but also because the existance of overlapping. Overlapping makes things even more complicated, 
	as the number of neighboring subdomains are increased from $4$ to $8$, with different overlapping patterns. To overcome this 
	problem, this paper proposes a irregular grid and overlapping friendly communication strategy that achieves optimal 
	minimum communication overhead, which is illustrated as follows.\\

%\begin{comment}
	\begin{figure}[htbp]
	  \centering
	  \includegraphics[width=0.5\textwidth]{communication_1_new.pdf}
	  \caption{Boundary nodes distribution}
	  \label{comm}
	\end{figure}
%\end{comment}

	Figure~\ref{comm} shows the boundary nodes' distribution along vertical direction. Solid blocks refer to subdomains without 
	overlapping, while dash blocks refer to subdomains with some extent of overlapping. Both outer boundary nodes and inter
	boundary nodes for subdomain 5 are marked in Figure~\ref{comm}. As we can see, the net connection can be irregular. On the
	other hand, as power grid is in multilayer structure, the inter and outer boundary nodes will appear at different positions
	among different layers. Besides, because of overlapping, there can be multiple solutions for a node's value at the boundary 
	overlapping area, as this node belongs to several subdomains at the same time. For convergence consideration, only one value 
	should be used to update subdomains. Here we use the node value from the subdomain that contains the node and has the biggest 
	index.\\

	From Figure~\ref{comm}, it can be clearly seen that for each subdomain, inter and outer boundary nodes' value 
	are the only information needed to be communicated during each iteration. Both types of nodes should be 
	sent/received to/from neighboring subdomains. Based on the geometrical characteristic of inter and outer boundary nodes, 
	each type of these nodes' value are stored in 8 node sets according to the geometrical relationship of a subdomain and its 
	neighboring subdomain. For example, ``south east" node set stores boundary nodes' value that are from south east neighboring 
	subdomain.
	As a result, for each subdomain, there are 16 node sets storing both types of boundary nodes' value. Boundary nodes within 
	these node sets are sorted in the same pattern, so that they are insensitive to multilayer and irregular structure.\\ 

	Each node will be
	indexed according to its ordering in the sorted boundary node sets. As the node sets are stored according to relative geometrical 
	relationship between 2 subdomains, for a node in
	a set along some direction, it can be located in the node set of neighboring subdomain with opposite direction. Besides, as
	the boundary node set are sorted in the same pattern, the node's index can be easily identified in the corresponding node set.
	For example, in Figure~\ref{comm}, inter boundary nodes along ``west north" 
	direction of subdomain 5 are also the outer boundary nodes along ``south east" direction of subdomain 7. Each node in the 
	2 node sets shares the same index. With this relationship, boundary nodes' value can be efficiently accessed.\\

	To minimize the start up time, we integrate the 16 node sets into local inter and outer boundary arrays. 
	Within each array, the 8 node sets will be stored in the increasing order of neighboring subdomains' index. 
	After each iteration, each subdomain has new solutions for all internal nodes. The inter 
	boundary nodes' value will then be copied into local inter boundary array. Processor 0 gathers this array from each subdomain
	into a global inter boundary array. Then, this array will be reordered to produce the global outer boundary array, which
	contains all the outer boundary nodes' value for each subdomain.\\ 

	The reordering process is shown in Figure~\ref{Fig4}. The index in global inter boundary array indicates the position for
	local inter boundary arrays for each subdomain. In Figure~\ref{Fig4}, the word within the local
	inter boundary arrays represents ``<subdomain\_id>\_<local\_ array\_ direction>".\\	

	For each subdomain, processor 0 scans its neighboring subdomains by increasing index and finds the local inter boundary 
	node arrays. According to the geometrical relation,
	these arrays are copied into the right position of global outer boundary array. For eg., for subdomain 5, the outer boundary
	arrays includes the ``north west" inter boundary array of subdomain 1, ``north" inter boundary array of subdomain 2, 
	and so on. The reordered global array will then be scattered to all processors, so that each processor has its outer 
	boundary nodes' value ready for later operation.\\ 
%\begin{comment} 
	\begin{figure}[htbp]
	  \centering
	  \includegraphics[width=0.45\textwidth]{reordering_new.pdf}
	  \caption{Reorder inter boundary array into outer boundary array}
	  \label{Fig4}
	\end{figure}
%\end{comment}

	After a subdomain receiving its local outer boundary array, the values will be assigned to corresponding internal nodes. As 
	mentioned before, nodes in overlapping area have several different values from overlapping subdomains. For these nodes, 
	as their values are assigned according to the ascending order of neighboring subdomain, even if there are several different
	values for one node, the new value will cover old one. In the end, the node's value is from the largest index neighboring 
	subdomain that contains this node.\\  
	
	In the above method, only one sending and one receiving operations are needed per iteration. As a result, the start up time 
	would only be $2t_s$, which is much less than the previously mentioned $16t_s$. In fact, it is also the optimal 
	minimum 
	start up time for this problem. On the other hand, time for transferring messages would be longer. But since the total number 
	of boundary nodes are always very small, combining with the fact that $t_w$ is very small, it can be concluded that the 
	time for transfering messages is almost negligible.\\ 

	In conclusion, the proposed strategy sucessfully achieves optimal munimum communication overhead. It  
	not only is insensitive to multiple layers or irregular connections, but also effectively overcomes the difficulty in 
	accessing right data, which is caused by overlapping.\\
	
  \subsection{Via detection technique}
	As mentioned before, power grid nodes are vertically connected by vias. If the vias are with no 
	resistance, in which case is called "shortened", those nodes always have identical voltage value. Since they all share the same
	solution, only stamping the representive
	node into matrix is a effective method to reduce the matrix size. In extreme case, if all the
	vias are shortened, then the 3D power grid can be modeled as a 2D grid without any loss of accuracy. The resulting matrix size
	is much smaller and can save a lot of time in simulation.\\ 
 
	It is also observed that if the vias connecting nodes have very small resistance value, it can be treated as "shortened" vias as
	well, since the difference between those nodes' voltage value is negligble. We set a threshold value for via nets. Via nets 
	whose values are below this threshold will be substitued as shortened nets. The above strategy can again be utilized to reduce 
	matrix size. Without loss of accuracy, this threshold value is experimentally set to be 1e-5.\\
 
  \subsection{Group technique} 
	Power grid usually contains several physically disjoint components, for example, VDD and GND circuit. With multi-voltage 
	technique, more components can be included, for example, VDD2 circuit. Since these components are totally independent, they can
	be solved in parallel. We utilize the group collective 
	communication technique of MPI. Processors are grouped into several clusters, where each cluster simulates one component. 
	If all the components are similar in size, runtime can instantly being saved by half or even more. 
	It should be noticed that this saving comes at a cost of using more processors. If $p$ processors are used to 
	solve different components sequentially, for a benchmark of $n$ components, the grouping technique requires $np$ processors to
	do the simulation. \\ 
 
\section{Numerical Experiments}
	We implement a direct LU solver: UMFPACK on a single CPU, and the proposed ASM on multi-core platform. 
	All codes are written in C++ and compiled with mpicxx under Linux system. 
  \subsection{Simulation Environment}
	The single CPU is with 2.67 GHz frequency and 24 GB RAM. The parallel platform is composed of a set of cluster 
	computing nodes, each with two 2.67 GHz Intel Xeon hex-core processors. Each node has in total 24 GB of RAM, with about 
	1.8 GB memory per processor. The primary network connecting the cluster machines is Voltaire QDR Infiniband. Message 
	Passing package MPICH2 is used.  
  \subsection{Implementation}
	Error tolerance is set to be $10^{-5}$ for all test cases to meet the following accuracy requirement:
	\begin{equation}
		Max(e) < 0.01 mv, Avg(e) < 0.1 mv\\ \label{eq_err}
	\end{equation}
	In the experiment, overlapping ratio is defined to be the ratio between overlapping length to the total side length of 
	a subdomain. To achieve a good balance between iteration numbers and matrix size, overlapping ratio is experimentally 
	setted to be 0.2.\\
			
	Three sets of benchmarks are used for comparison, including 
	industrial benchmarks ibmpg2 to ibmpg6, x250, x200, y250, y200 and 4 artifially produced benchmarks. ibmpg2 to ibmpg6 are 
	small size wire bond type benchmarks with vias, x250, x200 and y250, y200 are respectively two median C4 type benchmarks with
	shortened vias or regular vias. The 4 large scale benchmarks are all C4 type grids with vias. Simulation results are 
	listed in Table \ref{tb_v}, \ref{tb_s}, \ref{tb_m}, and \ref{tb_l}.\\

	\begin{table}[h]
	  \include{./tables/table_viagroup} \label{tb_v}
  	  \include {./tables/table_small} \label{tb_s}
  	  \include{./tables/table_median} \label{tb_m}
  	  \include{./tables/table_large} \label{tb_l}
	\end{table}
	
	In the above Tables, \#N refers to the total number of nodes of the ckt; \#B refers to number of blocks; \#C refers to number 
	of independent components in the 
	ckt; $t_c$ refers the simulation time for direct LU solver. $t_p$ refers the parallel simulation time with proposed method. ``-"
	means that the algorithm is out of memory. All $t_c$ is obtained by solving the circuits with independent components. All 
	simulation time includes both set up time and iteration time.\\

	Data in Table \ref{tb_s} shows the effectiveness of via detection technique and grouping technique. With via detection 
	technique, the 
	number of nodes which will be stamped into matrix is decreased by nearly 2/3. At the same time, grouping technique sucessfully
	further reduces parallel simulation time by half. If the 3 components are similar in size, even more simulation time can be
	saved.\\

	From Table \ref{tb_s}, \ref{tb_m}, the advantage of the proposed method is obvious. There is tremendous 
	speedup over direct LU solver. For each benchmark, there is at least tens of speedup. The biggest speedup is over 110X.\\ 
	%%%%% need to add more data nodes to show the limitation of over much processors.
	%If more processors can get allocated, it is possible to further decrease the runtime.\\

	Table \ref{tb_l} shows that the proposed method sucessfully finishes the simulation of large scale power grids
	in a short time, while LU can not even handle the problem size. In the experiment, the largest problem that LU can solve is 
	around 8M nodes per component.\\ 

	It should be noticed that since each processor in multi-core system has smaller memory size in hardware, grid needs to be 
	partitioned into enough subdomains so that each subdomain is small enough to be load in memory. As one processor handles one 
	subdomain, enough processors will be needed. For large scale benchmarks, if there is sufficient processor resource, several 
	components can be solved simultaneously. But if there is not enough processors, those components should be 
	solved sequentially. In Table \ref{tb_l}, netlist\_32M, netlist\_72M and netlist\_108M are 
	simulated with components being solved simultaneously. Each component consumes 500 processors. For netlist\_144M and 
	netlist\_192M, as not enough processors can get allocated, components are solved sequentially.\\

\section{Conclusion}
	This paper presents detail implementation parallel ASM for DC power grid analysis based on distribute memory machine. 
	A new strategy is proposed to efficiently handle the large problem size. Irregular and multilayer grid friendly overlapping 
	is introduced to accerelate the convergence. A new method is presented to achieve the minimum communication overhead. Via 
	detection and grouping technique are adopted to decrease the problem size, and introduce more parallelism. Experiments 
	results of several industrial benchmarks show that more than 110X speedup are gained over a state-of-art LU solver. It is also 
	the first time reported that benchmark as large as 192M can be processed and finished within 5 minutes. With enough processors, 
	even larger benchmarks can be simulated.

%\section*{Acknowledgment}

\bibliographystyle{abbrv}
\bibliography{sig-alternate}
\end{document}
