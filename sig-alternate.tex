\documentclass{sig-alternate}
\usepackage{enumerate}
\usepackage{verbatim}
\usepackage{graphicx}
\usepackage{subfigure}
\usepackage{rotating}
\usepackage{cite}
\usepackage{multirow}
\def\thesection{\arabic{section}}
\graphicspath{{./figures/}}

\begin{document}
\title{Parallel Power Grid Simulation with Multi Core Processors}

%\author{Ting Yu, Zigang Xiao, Martin.D.F. Wong}
%	{Depart of Electrical and Computer Engineering, UIUC\\
%	Champaign IL 61820\\
%	Email: tingyu1@illinois.edu, xxxxxxxxx, mfwong@illinois.edu}
%}
\maketitle

\begin{abstract}
	Because of technology, powergrid now are becoming more and more dense. There are usually multi-million number of nodes in
	industry powergrid benchmarks. As a result, analyzing power grid has become both computationnally expensive and memory 
	overconsuming. Parallel simulation is one of the most promising approaches to overcome these two main problems. Based on 
	multi-core platform, this paper presents an effecient parallel iterative Additive Schwarz Me-thod (ASM) for large-scale 
	power grid simulation. On one hand, a new strategy is proposed to address the large to very large scale problem size. Power 
	grid is divided into subdomains with different domain information being written into independent files. By reading corresponding 
	file, each processor only need to store data in one subdomain. Data load is balanced among processors and no processor
	need to spend memory to store the whole grid information. Though the I/O process when writing files is slow, the parallism 
	for reading files complements this cost. On the other hand, simulation time in parallel platform is mainly composed by 
	computation time and communication time. To reduce the computation time, 2D overlapping with a 9 point stencil pattern is 
	utilized to accelerate the convergence. Besides, via-detect techinique is presented to effectively reduce the subdomain 
	stamped matrix size, thus greatly help reducing simulation time. To reduce the communication overhead, a new idea is 
	proposed for boundary information exchange among processors. This strategy exibits optimal 
	communication overhead. Besides, in this paper, group technique is proposed to further increase parallism and reduce 
	simulation time. Experiment results on several industrial benchmarks show that the proposed method is able to achieve up to 
	more than 110X speedup over a state-of-the-art direct LU solver. Besides, it is the first time reported in this paper that 
	power grids contains over 190M nodes can be sucessfully solved only in less than 5 minutes.
\end{abstract}

\section{Introduction}
	For consideration of signal integrity, verification and further design optimization, powergrid analysis is one of the most 
	important step. To shorter the design period and cost, power grid simulation should be finished within a resonable time. 
	However, with the fast increase in scale of VLSI circuits, nowadays, power grid usually contains millions or even hundreds 
	of millions of nodes. The large problem size has made efficiently analyzing power grid a very chanllenging task by two aspects. 
	On one hand, it would be prohibitively expensive for traditional solvers such as 
	direct LU to solve the problem within a reasonable time. On the other hand, traditional methods may suffer a memory bottle 
	neck, which makes them uncapable to process the large or very large size problem. As a result, it is strongly required to 
	develop an efficient solver to address these problems.\\
%\begin{comment}
	\begin{figure}[htbp]
	  \subfigure[][2D view of a power grid]{
		\includegraphics[width=0.2\textwidth]{2D_pg_1_new.pdf}\label{subfig.1}}
	  \subfigure[][3D view of power grid]{
		\includegraphics[width=0.27\textwidth]{3D_pg_new.pdf} \label{subfig.2}}
	  \caption{Simple model of power grid: 
	  \subref{subfig.1} 2D view,
	  \subref{subfig.2} 3D view.}
	  \label{pg_model}
	\end{figure}

	Figure~\ref{pg_model} shows a two layer simplified power grid model without capacitors and inductances. Vias are used to 
	connect different layers. If transient analysis is performed, RLC model can be used.\\ 

	It is well known that analyzing the power grid equals to solving a linear system, where 
%\end{comment}
	\begin{equation}\label{eq1}
		Ax=b
	\end{equation}
	There are several serial simulation techniques developed to solve Equation \eqref{eq1}, including multigrid\cite{kozhaya}, 
	PCG\cite{Tsung-Hao}, random walks\cite{Boghrati}, domain-decomposition\cite{Quming}\cite{Zhongyu}, and so on. Random walks 
	is a stastical based method, and is only efficient for simulating small subset of nodes, not all the nodes in grid. 
	It will either be time consuming to solve all the nodes in grid, or will introduce inaccuracy in solution. The problems will
	deteriorate for large size power grid with only a few number of VDD/GND sources. Besides, when modeling with vias, this method
	can be trapped by vias, making the program hang forever. PCG method in\cite{Tsung-Hao} is known as the fastest linear solver 
	for large symmetric partial differential equations. But the total performance depends on the number of iterations. If the 
	iteration times is small, due to larger effort per iteration for PCG then other simpler methods such as Gauss-sidel, it may 
	not be a good choice. Besides, compared to GS, PCG consumes more memory, which will limit the maximum problem size that can be 
	solved.\\  	
	
	Multigrid method achieves near linear solve time with input size. The fine grid is recursively restricted into coarse ones 
	and the solution from coarse grids will be interpolated back into fine ones. There are geometric based multigrid method as well as 
	algebraic based multigrid method. For both methods, because of restriction and interpolation, only approximate solution can be 
	obtained. The error can be intractable for large power grids. Domain decomposition is based on idea of "divide and conquer". 
	There are two kinds of domain decomposition(DD): Schur Complement DD and Additive Schwarz DD. The main difference is that Shur 
	Complement DD partitions find grid into subdomains
	and a common interface, while in Additive Schwarz DD, there is no interface between subdomains. Overlapping can be introduced to
	Additive Schwarz DD when partitioning, for example\cite{Zhongyu}. The limitation of Schur Complement DD is that because of the 
	interface subdomain, there is a dense schur matrix. The dense characteristic limits the interface size, and thus decides Schur 
	Complement DD can not solve too large a problem.\\

	Considering the fact that power grid is usually in large size, even for an efficient serial solver, more and more time would be
	spent for simulation. When the input size is above some scale, the simulation time would be inprohibitely expensive.
	Besides, serial algorithms do not have enough memory to hold the very large scale grid size. On the other hand, parallelism is 
	possible for some algorithm, and can thus bring a lot of speed up. Considering as well that architectures as multi-core platform 
	can potentially overcome the memory issue, parallel technique is a promising tendancy to realize a good simulator.\\ 

	There are already several published work about analyzing power grid with parallel methods, including multigrid on GPU\cite{Zhuofeng}, parallel direct LU 
	solver\cite{Super_LU_website}, public domain parallel ASM package\cite{PETSC_website}, and parallel domain decomposition
	\cite{kaisun}\cite{voronov}. For \cite{Zhuofeng}, though it is very fast, the above mentioned problems for multigrid still exist. 
	If the system is ill-conditioned, this method can even have convergence problem. Besides, as mentioned before, GPU platform 
	can not handle very large scale power grid problem.\\
 
	SuperLU\_MT and SuperLU\_DIST are two parallel direct LU solver developed in \cite{Super_LU_website}. SuperLU\_MT is 
	based on shared memory machine. Limited by the number of threads of shared memory machine, not too much speedup can be gained. 
	It is reported to be only 5 to 10 folder faster than commecially popular solvers\cite{Super_LU_website}. Besides, the largest 
	problem this software can solve is limited by the system memory of a single processor/core. SuperLU\_DIST is developed on 
	distribute 
	memory machine. A single processor is used to first read in the matrix, then distribute the matrix to other processors. 
	There is a obvious disadvantage for this strategy: as the whole matrix is stored 
	in a single processor, it will be a heavy load for large problem size. As a result, the problem this package can 
	process can not be very large as well. The largest problem size that is reported for this package is 2 million nodes
	\cite{Super_LU_website}.\\

	Domain decomposition is theoretically suitble for parallization, as there are many independent subdomains which can be
	solved by different processors simultaneously. Public domain package PETSC\cite{PETSC_website} implements ASM in parallel.
	However, there are several limitations. First, PETSC can only handle regular and uniform grid. Thought overlapping is 
	used, but the overlapping amount is counted as the times of pitches, which is not the case for irregular connected grid. 
	Besides, ASM package in PETSC is only fit for 2D grid. Another problem is that since a global matrix is still
	needed to be stored somewhere, it is not fit for large size problem. Parallel domain decomposition in \cite{kaisun}\cite{voronov} 
	are implementations of SCM or ASM. Because of solving dense matrix or storing global matrix, the problem they can solve can not 
	be very large.\\

	This paper propose an efficient parallel implementation of ASM, which can not only process large to very large scale problem size, 
	but also achieves much speedup to a state-of-art direct LU solver. The main contributions include:\\
	\begin{enumerate}[1)]
	\item Based on distributed memory machine, a new data storage strategy is proposed, which can overcome the uneven data 
	load problem of SuperLU\_DIST and PETSC. The new method can thus handle large to very large-scale problem.
	\item Fully irregular grid friendly 2D overlapping with 9 point stencil pattern is implemented to help convergence. 
	\item A new data communication mechanism is proposed, which achieves the optimal minimum communication overhead.
	\item Via detect technique is presented to efficiently reduce matrix size without loss of accuracy.
	\item A new group techinique is introduced to further increase parallism.\\
	\end{enumerate}

	The proposed method can process irregular connected power grid. With the above mentioned techniques, it can not only solve 
	power grid accurately and efficiently, but also can handle large to very large problem. In conclusion, the proposed method 
	is able to effectively meet both two key requirement of power grid analysis.\\

	The rest of this paper is organized as follows. Section 2 gives an overview of ASM. Section 3 prestents 
	the parallel implementation of ASM based on a distributed memory machine. Experimental results are illustrated in section 4. 
	Finally, conclusion is given in section 6.
%\begin{comment} 

\section{Additive Schwarz method(ASM) overview}	
	In mathematics, the additive Schwarz method, named after Hermann Schwarz, solves a boundary value problem for a partial 
	differential equation approximately by splitting it into boundary value problems on smaller domains and adding the results.
	Given a symbolic 2D view of a regular power grid such as Figure~\ref{Fig1}, Figure~\ref{Fig2} shows two types of ASM. 
	Figure~\ref{Fig2}\subref{subfig.1} shows the algebraic ASM method utilized in \cite{kaisun}. The global matrix is built first. 
	Then, it will be partitioned into several subdomains in row direction. 
	Overlap can be introduced at this stage, which can be seen in Figure~\ref{Fig2}\subref{subfig.1}. The partitioned subdomains
	are then updated either with Jacobi or Gauss-sidel iteration.\\
	\begin{figure}[htbp]
	  \centering
	  \includegraphics[width=0.25\textwidth]{2D_grid_new.pdf}
	  \caption{2D view of a regular power grid}
	  \label{Fig1}
	\end{figure}
	
	\begin{figure}[htbp]
	  \subfigure[][Algebraic ASM]{
		\includegraphics[width=0.22\textwidth]{AASM_new.pdf}\label{subfig.1}}
	  \subfigure[][Geometric ASM]{
		\includegraphics[width=0.22\textwidth]{GASM_new.pdf} \label{subfig.2}}
	  \caption{Two types of ASMs: 
	  \subref{subfig.1} Algebraic ASM,
	  \subref{subfig.2} Geometric ASM.}
	  \label{Fig2}
	\end{figure}

	Figure~\ref{Fig2}\subref{subfig.2} illustrates the geomegric ASM utilized in \cite{Zhongyu}. The grid is first partitioned
	into several subdomains. Each subdomain then forms local matrix and can be solved either by direct solvers or iterative solvers. 
	Jacobi or Gauss-sidel method is used to update subdomains. Overlap can be introduced during the partitioning stage.
	Different from algebric ASM, where overlapping can only happens between a subdomain and one neighbor subdomain, in geometric ASM, 
	fully overlapping can now happens between one subdomain and all its 8 neighboring subdomains, as is seen from Figure~\ref{Fig2}
	\subref{subfig.2}. For the rest of this paper, this overlapping pattern is called 9 point stencil pattern.\\
	
%\end{comment}	
	It is pointed out that introducing overlaping between different subdomains leads to faster convergence, 
	as overlapping reduces the matrix eigenspectrum\cite{Klawonn}\cite{Taopeng}. The more overlapping, the faster convergence,
	but the equivalent subdomain size will  be increased. As a result, there is a balance point for overlap ratio to gain the 
	optimal total runtime. SOR, PCG, MG methods can be combined with ASM to accelerate convergence. However, the resulting methods
	are not fit for parallization. The reasons are as follows: SOR method is not 
	stable, as the optimal $w$ value depends on system matrix eigenvalues. It is hard to choose the optimal $w$ for large
	size problem, a bad $w$ value can even lead to divergence; MG may have
	error or even convergence problems, especially for 3D irregular grid; In PCG, a global
	matrix is required, which limits the maximum problem size that can be solved.\\ 
	 
	There are already works talk about parallization of overlapping ASM, such as\cite{kaisun}. One problem in \cite{kaisun} is 
	that the
	data load is not evenly distributed. Also, the communication strategy is not trival, with the cost of gathering all the
	nodes value in the grid and bcast it to all processors at each iteration. Since only boundary nodes values are essentially all 
	needed for processors to update boudary value, and the number of boundary nodes is only a very tiny ratio of the whole grids' 
	nodes, it would be a waste to gathering all nodes' value each iteratoin. The ideal case would be 
	each processor only sends and receives only its corresponding boundary nodes' values from/to its neighboring subdomains. 
	However, because of overlapping and irregular grid structure, it is a challenging problem to realize this idea, which is
	illustrated in section 3.\\
 
\section{Parallel Implementation On \\Multi-core Processors}
  \subsection{Parallel platform}
	There are two types of multi-core architecture: shared memory and distributed memory machine. In shared memory machine, as 
	multiple processes share common address space accessible to all, there is no overhead for data communication. However, limited
	by number of threads and memory size, it is not suitable for the problem. On the other hand, distributed 
	memory machine can be composed by a lot of processors, while each processor has independent memory. The cost is that extra time 
	is needed for data communication, as each processor only have accessibility to its local data. Considering the large problem 
	size and efficient simulation time requirement, distributed memory machine would be a better choice. The implementation is based
	on widely utilized message-passing library(MPI).
  \subsection{Partitioning}
	In \cite{Zhongyu}, 1D partitioning is thought to be in better performance than 2D partitioning. However, the comparison is not 
	fair, as the number of elements in 1D partitioned subdomain and 2D partitioned subdomain are not equal. Here we illustrates that
	for parallel consideration, 2D partitioning strategy presents a better performance than 1D partitioning. It is because 
	partitioning across more dimensions yields more neighboring subdomains but smaller total volume of communication than 
	partitioning across fewer dimensions, which is shown in the following example.\\

%\begin{comment}	
	Suppose the power grid in Figure~\ref{Fig1} is partitioned in 1D subdomains and 2D subdomains, which is shown in 
	Figure~\ref{Fig3}.
	\begin{figure}[htbp]
	  \subfigure[][1D partitioning]{
		\includegraphics[width=0.22\textwidth]{1D_partition_new.pdf}\label{subfig.3.1}}
	  \subfigure[][2D partitioning]{
		\includegraphics[width=0.22\textwidth]{2D_partition_new.pdf}\label{subfig.3.2}}
	  \caption{2 types of partitioning:
	  \subref{subfig.3.1} 1D partitioning
	  \subref{subfig.3.2} 2D partitioning }
	  \label{Fig3}
	\end{figure}

	For fairness, in the above two types of partitioning, each subdomain in 1D partition and 2D partition has the same number
	of nodes. As a result, the total number of subdomains for 1D partitioning and 2D partitioning is the same. For each subdomain 
	in Figure~\ref{Fig3}\subref{subfig.3.2}, suppose it is square and each side is contains $k$ nodes. As a result, the total 
	number of nodes needed to be interchanged with neighboring subdomains would be $4k$. On the other hand, as 
	there are $m_2$ subdomains along vertical direction, for a subdomain in Figure~\ref{Fig3}\subref{subfig.3.1}, each side contains
	$m_2\times k$ nodes. As a result, there are $2\times m_2\times k$ nodes needed to be interchanged with its left and right 
	neighboring subdomains. Equation\eqref{eq2} illustrates the relationship of the communication load between these two types of
	partitioning. As $m_2>2$ is generally always true, 2D partitioning thus have more favorable surface-to-volume ratio.    
	\begin{equation}
		2\times m_2\times k > 4\times k, \hspace{0.5 in} m_2 > 2 \label{eq2}
	\end{equation}
%\end{comment}
	Besides, considering parallel performance, 2D partitioning presents a better scalability than 1D partitioning. This can be
	illustrated from the isoefficiency expressions of 2 type partitoning. Isoefficiency represents the scalability of algorithm
	with input problem size. The lower order in isoefficiency expression, the better scalabilify. Isoefficiency function for 1D 
	and 2D partitioning is respectively $\Theta(p^2)$ and $\Theta(p)$\cite{Grama}\cite{Kumar}, where p is the number of processors. 
	It can seen that for given number of processors, for 2D partitioning, far smaller problem is required to achieve given 
	efficiency than with 1D partitioning.\\

	Combining the two considerations, 2D partitioning is chosen in this paper.\\	
	  
  \subsection{Data storage}
	In parallel program, it is very important to keep data balance among processors. This is not only for computing
	consideration, but also for memory storage consideration. In \cite{Quming}, input data is first loaded into a single processor 
	and distribute to others. By doing this, the single processor will spend a lot of extra memory for storing the data. As a result,
	the problem size can be solved will be limited. In this paper, we propose a new data distributing method, so that the amount of
	data being loaded for each processor is similar. This strategy effectively eliminates the limitation of problem size.\\

	According to the number of subdomains and the grid's geometrical information, geometrical boundary information for each
	subdomain can be extracted. With this geometrical information, input netlists can be concluded into different files.
	Each subdomain has a indepent file which contains all the nets information within it to build up its local matrix. 
	This stage is done by a single processor. Then, all processors will parse data from corresponding file simultaneously, and build 
	up local matrix. By doing this, processors does not need to store data that belongs to other subdomains, thus no memory is 
	wasted for storing irrelevant data. \\

	In the above process, though I/O process of writing files is slow, the parsing stage, which previously occupys most part of the 
	time doing by a single processor, can now be executed by all the processors in parallel. As a result, there will not be extra
	overhead with this strategy. Simulation result in Table[XXX] proves this assumption.\\
	 
       \subsection{Communication overhead}
	After each subdomain gets its local solution, before next iteration, boundary nodes value needs to be collected from its 
	neighboring subdomains to update
	the right hand side. In the rest of this paper, these nodes are defined as outer boundary nodes. Also, the internal nodes' 
	values which are directly connected to those boundary nodes need to be sent out
	to corresponding neighboring subdomains to update the right hand side of those subdomains. These nodes are defined as internal
	boundary nodes. An intuitive strategy for sending and collecting boundary informations would be as follows:\\

	\begin{enumerate}[1.]
	  \item A subdomain sends its internal boundary nodes values sequentially to 8 neighboring subdomains;
	  \item The subdomain  then collects its outer boundary nodes sequantially from 8 neighboring subdomains.\\
	\end{enumerate}
	
 	However, this strategy is not efficient in message passing parallel system. To analyze the reason, we can first peer the simple
	model for time required to send message between adjacent nodes:
	\begin{equation}
		T_{msg} = t_s + t_wL\label{eq_msg}
	\end{equation}
	where $t_s$ is the start up time, $t_w$ is the incremental transfer time per word, and $L$ is the length of message in words.\\
	
	For most real parallel system, $t_s$ is tens of thousands larger than $t_w$. As a result, it is highly perferable to send less 
	large messages rather than many smalle messages to reduce the start up time. Since the total number of boundary nodes is small, 
	time spend on $t_wL$ is short. On the other hand, for the above mentioned strategy, $8\times 2 = 
	16$ times send or receive operations is needed for a subdomain per iteration. As a result, in total $16\times t_s$ would be used 
	for start up per iteration. As there may be a lot of iterations before reaching convergence, the total start up time
	should not be ignored. If in each iteration, all communications can be finished in just 1 or 2 operations, minimum start up time 
	can be achieved, which can greatly decrease the simulation time.
	Besides, another problem with the above strategy is that the existance of overlapping makes control flow complicated. Extra work 
	is needed to locate the index of 4 diagonal processors. The proposed method in this paper can achieve optimal minimum 
	communication overhead with a simple control flow.\\

%\begin{comment}
	\begin{figure}[htbp]
	  \centering
	  \includegraphics[width=0.4\textwidth]{communication_1_new.pdf}
	  \caption{Boundary nodes distribution}
	  \label{comm}
	\end{figure}
%\end{comment}

	Figure~\ref{comm} shows the boundary nodes distribution for communication. Solid grid refers to subdomains without overlapping, 
	and dash line refers to subdomains with some extent of overlapping. The
	top vertical short line with a node highlighted refers the vertical boundary net between subdomain 5 and subdomain 7-9, where the
	highlighted nodes are outer boundary node set for subdomain 5. The bottom 
	vertical short line with a node highlighted refers the boundary nets between subdomain 7-9 to subdomain 5, where the highlighted 
	nodes are internal boundary node set for subdomain 5. The internal boundary node set are also the outer boundary node set for 
	subdomain 7-9. As we can see, the net connection can be irregular. Because of multilayer structure of
	power grid, the two sets of nodes will appear at different layers either at same position or different position. At the
	same time, because of the overlapping, there can be multiple values for a node at the boundary overlapping area. For example, 
	the top node set connected with red short lines belongs to subdomain 7 and 8. Only one value should be used for the update of 
	subdomain 5. Here we use the node values from the latest subdomain.\\

	From Figure~\ref{comm}, it can be clearly seen that for each subdomain, internal and outer boundary nodes' values 
	are the only information needed to be communicated among processors during each iteration. Each set of nodes should be 
	sent/received to/from different neighboring subdomains, and nodes in the overlapping area should be sent/received to/from  
	overlapping subdomains. Based on the geometrical characstic of the internal and outer boundary nodes, they can respectively 
	be stored in 8 node sets along 8 directions, for example south east and so on. These directions represents geometrical 
	relationship between the subdomain and its neighboring subdomains. All 16 node sets for each subdomain will be sorted in the 
	same pattern to make them insensitive to multilayer and irregularity structure.\\ 

	Each node will be
	indexed according to its ordering in the sorted node sets. As all the 16 node sets are soreted in the same pattern, for a node in
	a set along some direction, the index will be identical with that in the other node set of neighboring subdomain with opposite
	direction. For example, subdomain 5 in Figure~\ref{comm}, the nodes of internal boundary node set along west north 
	direction are the ones of outer boundary node set along south east direction in subdomain 7. Each node in the 2 node sets 
	shares the same index. According to the direction relationship, one type of boundary nodes' value can be correctly accessed by 
	the other type of boundary nodes in neighboring subdomain.\\

	To minimize the start up time, we integrate the 16 node sets into internal and outer boundary arrays. 
	Within each array, the 8 node sets will be stored in the increasing order of neighboring subdomains' index. 
	For example, in Figure~\ref{comm}, for subdomain 5, the order is from subdomain 1 to 9 
	except subdomain 5 itself. After each iteration, each subdomain has new solution for all internal nodes. The internal 
	boundary nodes' value will be first copied into internal boundary array. Processor 0 will gather this array from each subdomain
	into the global internal boundary array. Then, processor 0 will reorder this array to produce outer boundary array 
	for all subdomains. The reordering process is as follows: \\

	For each subdomain, processor0 finds its neighboring subdomains by increasing index. It finds the internal boundary 
	nodes' values for each neighboring subdomain, which is exactly the outer boundary nodes' values of the aim subdomain. Direction 
	relationship is utilized to order the 8 internal boundary node set values into the global outer boudnary array. The 
	global array will then be scattered from processor 0 to all processors, so that each processor has its outer boundary nodes'
	values from 8 neighboring subdomains.\\ 
%\begin{comment} 
	\begin{figure}[htbp]
	  \centering
	  \includegraphics[width=0.45\textwidth]{reordering_new.pdf}
	  \caption{Reorder internal boundary array into outer boundary array}
	  \label{Fig4}
	\end{figure}
%\end{comment}

	After a subdomain receiving its local outer boundary array, the values will be assigned to corresponding internal nodes by 
	index. As mentioned before, in overlapping area, a outer boundary node for a subdomain can be an internal boundary node for 
	several neighboring subdomains. For these nodes, as the outer boundary nodes' value are accessed by neighboring subdomains 
	in the order of increasing index, the old value of the same node is covered by new ones, and the newest value
	comes from the largest index neighboring subdomain that contains this node. As a result, the unity of boudary nodes' value 
	is kept.\\  
	
	With these two arrays, only one sending and one receiving are needed. As a result, the start up time would only be $2\times 
	t_s$, which is much less than the previously mentioned $16\times t_s$. It is also the optimal minimum start up time. On the other
	hand, considering the time for transferring messages, which is $t_wL$ in Equation\eqref{eq_msg}, the message length is longer.
	But since the total number of boundary nodes will always be very small, combining with the fact that $t_w$ is very small, 
	it can be concluded that the time for transfering message is almost negligible.\\ 

	Conclude from above process, the proposed strategy sucessfully achieves optimal munimum communication overhead. It  
	not only is insensitive to multiple layers or irregular connections, but also effectively overcomes the difficulty of 
	accessing right data introduced by overlapping.\\
	
  \subsection{Via-detect technique}
	As mentioned before, power grid nodes is vertically connected by vias. If the vias are with very small resistance, or no 
	resistance, in which case is called "shortened", those nodes always have identical voltage value. In order to decrease matrix
	size, only one representative node should be stamped into matrix. In extreme case, if all the
	vias are shortened, then the 3D power grid can be modeled as a 2D grid without any loss of accuracy. The resulting much smaller 
	matrix size will make the simulation much more faster.\\ 
 
	The above idea is realized by comparing via values with a threshold value. When parsing the input
	via nets, those via nets whose value is below the threshold will be substitued as a shortened net. After reading all the 
	input nets, all nets will be rescanned and representative nodes will be extracted from shortened nets to be stamped into matrix. 
	Without loss of accuracy, this threshold value is experimentally set to be 1e-5.\\
 
  \subsection{Group technique} 
	Power grid is usually composed with several physically disjoint components, for example,
	VDD and GND. If multiple voltage technique is used, more components can be included. These component can be solved independently. 
	For a single processor, they will be solved sequentially. However, in distributed multicore machine, 
	with enough processors, they can be solved simultaneously. MPI provides group collective communication technique, which makes 
	it possible. Processors are grouped by component order, and all the communication of a component circuit happens only among 
	the processors of the corresponding group. Suppose all the components have similar size, combined with above paralle program, 
	group technique instantly cuts the runtime by half or even more. It should be noticed that group technique achieves less runtime
	at a cost of more processors. If $p$ processors are used to parallel the simulation, with a benchmark of $n$ components, group 
	technique will consume $n\times p$ processors. \\ 
 
\section{Numerical Experiments}
	We implemented the direct LU solver with UMFPACK on a single CPU, and parallel the proposed ASM on multi-processor platform. 
	All codes are written in C++ and compiled with mpicxx under Linux system. 
  \subsection{Simulation Environment}
	The single CPU is with 2.67 GHz frequency and 24 GB RAM. The parallel platform is composed of a set of cluster 
	computing nodes, each with two 2.67 GHz Intel Xeon hex-core processors. Each node has in total 24 GB of RAM, with about 
	1.8 GB memory per processor. The primary network connecting the cluster machines is Voltaire QDR Infiniband. Message 
	Passing package MPICH2 is used.  
  \subsection{Implementation}
	Error tolerance is set to be $10^{-5}$ for all cases to meet the following accuracy requirement:
	\begin{equation}
		Max(e) < 0.01 mv, Avg(e) < 0.1 mv\\ \label{eq_err}
	\end{equation}
	In the experiment, overlap ratio is defined the ratio between overlap length on one end to the total side length of a subdomain. 
	To achieve a good balance between iteration numbers and matrix size, overlapping ratio is experimentally 
	setted to be 0.2.\\
			
	Three sets of benchmarks are used for comparison, including 
	industrial benchmarks ibmpg2 to ibmpg6, x250, x200, y250, y200 and 4 artifially produced benchmarks. ibmpg2 to ibmpg6 are 
	small size wire bond type benchmarks with vias, x250, x200 and y250, y200 are respectively two median C4 type benchmarks with
	shortened vias or regular vias. The 4 large scale benchmarks are all C4 type grids with vias. The simulation results are 
	listed in Table \ref{tb_v}, \ref{tb_s}, \ref{tb_m}, and \ref{tb_l}.\\

	\begin{table}[h]
	  \include{./tables/table_viagroup} \label{tb_v}
  	  \include {./tables/table_small} \label{tb_s}
  	  \include{./tables/table_median} \label{tb_m}
  	  \include{./tables/table_large} \label{tb_l}
	\end{table}
	
	In the above Tables, \#N refers to the total number of nodes of the ckt; \#B refers to number of blocks; \#C refers to number 
	of independent components in the 
	ckt; $t_c$ refers the simulation time for direct LU solver. $t_p$ refers the parallel simulation time with proposed method. "-"
	means that the algorithm is out of memory. All $t_c$ is obtained by solving the circuits with independent components. All 
	simulation time includes both set up time and iteration time.\\

	Data in Table \ref{tb_s} shows the effectiveness of via detect technique and group technique. With via detect techniqie, the 
	number of nodes which will be stamped into matrix is decreased by nearly 2/3. At the same time, group technique sucessfully
	further reduces parallel simulation time by half. If the 3 components are similar in size, even more simulation time can be
	saved.\\

	From Table \ref{tb_s}, \ref{tb_m}, the advantage of the proposed method is obviously, which shows tremendous 
	speedup over direct LU solver. For each benchmark, there is at least tens of speedup, and the biggest speedup is over 110X. 
	%%%%% need to add more data nodes to show the limitation of over much processors.
	If more processors can get allocated, it is possible to further decrease the runtime.\\

	Table \ref{tb_l} shows that multi-core platform sucessfully finished the simulation of large to very large scale power grids
	in very short time, while LU can not even handle the problem. In the experiment, the largest problem that LU can solve is 
	around 8M nodes per component.\\ 

	It should be noticed that since each processor in multi-core system has smaller memory size in hardware, grid needs to be 
	partitioned into enough subdomains so that each subdomain is small enough to be hold in memory. As one processor handles one 
	subdomain, enough processors will be needed. For large scale benchmarks, if there is sufficient processor resources, there is
	no problem to solve several components simultaneously. But if there is not enough processors, those components should be 
	solved sequentially to save processor consumption. In Table \ref{tb_l}, netlist\_32M, netlist\_72M and netlist\_108M are 
	simulated with components being solved simultaneously. Each component consumes 500 processors. For netlist\_144M and 
	netlist\_192M, as not enough processors can get allocated, components are solved sequentially.\\

\section{Conclusion}
	The conclusion goes here\cite{package}.

\section*{Acknowledgment}

\bibliographystyle{abbrv}
\bibliography{sig-alternate}
\end{document}
